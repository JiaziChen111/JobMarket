\documentclass{econtex}
\usepackage{JobMarket}
\usepackage{econtexSetup}
\usepackage[normalem]{ulem}
\newcommand\redout{\bgroup\markoverwith
{\textcolor{red}{\rule[.5ex]{2pt}{1pt}}}\ULon}

\pagestyle{plain}

\begin{document}
\hfill{\tiny \jobname, \today}
\vspace{.1in}

\begin{verbatimwrite}{\jobname.title}
Recommendation Letters Procedures
\end{verbatimwrite}
\centerline{\Large Recommendation Letters Procedures}\medskip\medskip

\centerline{\today}\medskip\medskip

\ifdvi\large\fi

\section{Information for Students and Staff}

The department will send out recommendation letters in these ways:
\begin{enumerate}
\item \EJM ~ (\EJMLink)
\item \AJO ~ (\AJOLink)
\item \AEA ~ (\AEALink)
\item \Interfolio ~ (\InterfolioLink)
\item ``Pull'' Email (that is, when the employer, after receiving your application, sends an email to your recommenders asking them to upload the letters to a special-purpose site that the employer has set up)
\item ``Push'' Email (that is, when the employer provides an email address to which letters should be sent)
%\item Snail Mail
\end{enumerate}

All of these methods require that your recommenders send PDFs of their letters 
to~\JMStaffEmail.\footnote{A few employers may demand that the PDFs be ``digitally signed.''  For now, we will ignore this demand.  If and when those employers start to make up a critical mass, we will have to figure out how to train the faculty in how to make digitally signed letters.}  However, you should NOT ask them to send their letters until you have done the things that YOU are supposed to do (detailed below).

Note that these options do NOT include a method of getting letters to
any employer that has set up its own database system and wants
recommenders and/or students to register for a username, password,
etc.  In practice, such places will invariably accept letters sent by
email to some department staff person, and that is what we will do.
YOU need to find out the email address of that staff person.  If there
are any employers that absolutely insist that applicants and
recommenders learn to use their own unique systems, letters to those
employers will be completed only AFTER ALL OTHER LETTERS FOR ALL OTHER
STUDENTS have been sent using the preferred methods listed above.
This is vitally important because from past experience we know that
the confusion and delays caused by proprietary systems have the
potential to end up making everyone's letters late, so we can do this
only once that risk has been eliminated.  The only exception to this
rule is the IMF, which has its own procedures that we do follow.

Some employers have a deadline before Nov 15 (a few as early as Nov
1).  Most employers have deadlines in middle to late November.
Finally, employers with late job postings may appear in the December
JOE.

For each timeframe, you will send to \JMStaffEmail~a list of employers
to which you are applying: \EM-\texttt{[Which]}~where ``[Which]'' will
be the words ``Early,'' ``Middle'', or ``Late.''  There should be at 
least a week between the date when you send the spreadsheet and the
earliest application due-date contained therein.  Thus, if you want to
apply to some jobs that have a deadline of Nov 1, then by October 23
you would send \EM-\Early.  It would be good to include also in the
``Early'' spreadsheet employers with a deadline up to Nov 15 or 16, 
so that the workload is spread out over time as much as possible.  
By Nov 8 you would send \EM-\Mid~with the employers
who have November deadlines that you did not send in your ``Early''
spreadsheet.  Finally, if any new jobs are posted in the December JOE
that you want to apply to, then sometime around Dec 7 you would send
\EM-\Late.  (It is good to include as many employers as possible in
your ``Late'' spreadsheet because this spreads out the work that
\JMStaff~has to do more evenly over time).

You will get these 3 (and ONLY these 3) opportunities to apply to
jobs.  It is impossible for the staff, for every student, to keep
track of more than 3 spreadsheets each of which has multiple
employers; experience tells us that everything breaks down if every
few days each student drops by the office (or sends an email) saying
``oh by the way, please add the following 4 employers to my list.''

Broadly, for each iteration your steps are as follows:
\begin{enumerate}
\begin{comment}
\item Create a manila folder with your name nicely written at the top
  (last name, first name, and moniker) in which the staff will keep all your
  records during the job market process.  (Ask the staff to give
  you the folder; they will probably want them all to be the same
  size/style).

\item Produce a list of your recommenders' names (3 or 4, usually) ON
  PAPER which you will give to the staff so that they can tick off
  recommenders' names as the recommenders' letters are received.
  (This checklist will be the first item that goes in your manila folder.)
\end{comment}

\item Produce your version of \EM~(by ``your version'' I mean, of
  course, to rename the template file to, for example,
  \texttt{EmployersThomK-Early.xlsx} if you are Kevin Thom and it is
  your \Early~list) that contains {\it all} the employers, and {\it
    only} the employers, that you are actually applying to in this
  round.  (You probably will have some employers you have thought
  about but have decided not to apply to, or to apply to in later rounds; if you want to preserve that
  information, please copy and paste it to another spreadsheet, and
  remove it from your main \EMW~spreadsheet that you will give to the
  staff.)

\item Sort your \EM~spreadsheet according to five sort keys:
 \begin{enumerate} 
 \item Date you asked the staff to send the recommendation letters; 
 \item Method of recommendation (\AEA, \AJO, \EJM, \Interfolio, push email (for sorting purposes,
  label it ``email-push''), pull email (label it ``email-pull''), or
  ``other'' with the understanding that anything in ``other'' will be
  sent only after ALL other recommendations for ALL other students are
  sent);
  \item Academic vs.\ Nonacademic;
  \item Domestic vs.\ Foreign;
  \item Employer Name. 
  \end{enumerate} 

  (There should be a macro (\texttt{Ctrl}+\texttt{w}) built into \EM~that can be
  executed to accomplish this sort -- see the instructions in
  \EM\texttt{-Instructions.xlsx}).

%\item \redout{After sorting the spreadsheet, execute the macro to ``hide'' the information that \JMStaff~does not need for purposes of tracking which letters have been sent.  (Instructions for how to execute this macro should be on the first page of \EM).}

\item Then send \EM~to \JMStaffEmail, and be sure to cc \JMPOEmail.
  (The placement director needs to know where you have applied for a
  host of reasons, including being prepared for calls that employers
  might make seeking further info).

%\item \redout{You'll probably want next to immediately ``unhide'' those columns that were hidden, so that you can see them yourself.  Again, there should be a macro that accomplishes this described in the first-page instructions.}

\item {\it After having sent} their \EMW~spreadsheet, the student
  should follow the steps outlined in detail below for those
  applications that involve either push or pull email.

\item {\it After taking care of their push and pull email applications}, students should complete their \EJM, \AJO, \AEA, and \Interfolio~applications for this round (see below for details).

\item {\it After having done all of this}, the student should email
  their recommenders and ask them to send their letters to
  \JMStaffEmail.  Note that your request to recommenders needs to come
  last because if the letters arrive before the other steps have been
  taken, the letters may get lost.  (Though it would be wise to remind
  your advisors a week or two beforehand that letters from them will
  be needed soon).

\end{enumerate}

More detailed procedures are below:
\begin{enumerate}

\item {\bf \EJM ~ (\EJMLink):}
\begin{itemize}
\item For JHU faculty recommenders, \EJM~identifies recommenders using
  an official JHU email addresses.  

\begin{itemize}
\item Many faculty already have an~\EJM~account; for these
  recommenders, it will be obvious how to make~\EJM~send the
  recommender a request for a letter.

\item Some faculty members may not yet have an~\EJM~account.  
For security reasons, \EJM~now insists that each
  recommender can have only one account at \EJM, associated with a
  unique \texttt{@jhu.edu} email address.  Even though the faculty member may never use
  that email address (instead, for example, using a \texttt{gmail}
  account for all correspondence), they nevertheless do {\it have} a
  JHU email address.  If they don't know what it is, they can ask
  Nina.  You need to find out from the faculty member what email
  address they want you to give to \EJM~when it asks for their address.  Once you
  give~\EJM~that email address, a message will be sent to the email
  address informing the faculty member that an account has been
  established in their name at~\EJM.  After logging in, they should
  designate \JMStaffEmail~as a ``proxy'' who can load their letters
  for them, and so when you request subsequent letters the request should
  actually go to \JMStaffEmail, who will receive a ``pull'' email
  notification and has to click on the link provided for each
  recommender, click on the name of the student who required the
  letter, upload the letter, and select all the employers.

\end{itemize}


\item Recommenders who are not Department of Economics full-time faculty and who do NOT have their own
login ID at \EJM~will need to have an account created for them (the account is created automatically the first time a student identifies the recommender by giving \EJM~the recommender's email address), then they must upload their letters themselves. \footnote{They follow the same steps that the
    staff completes in the bullet point about JHU faculty.  They
    log in, they upload their letter, they indicate that it's for, e.g.,
    \texttt{colleen.carey@jhu.edu}, and then they select all employers and tick the box for all
    future employers.}
New \EJM~security measures prevent us from uploading letters on behalf of people
who are not JHU faculty.  The student must communicate this information to
  the recommender.

\end{itemize}

\item {\bf \AJO ~ (\AJOLink):}
Students will enter Maggie's name, with the email address \JMStaffEmail, on the cover sheet provided by the AJO system. Check the box: ``must check here if the person above will upload letters on behalf of multiple writers'' and enter the actual writers' names in the box provided. After all reference letters for each student are received, Maggie will make 1 PDF file of all letters and upload their file to the AJO site. The student will be allowed to see when this upload is complete. This service will be provided for both JHU Economics and external recommenders.

\item {\bf \AEA ~ (\AEALink):}
Students who are using \AEA~must ask their JHU Economics Department reference writer to go to \AEARecLink~ and set up a surrogate for their reference letters. The surrogate name is Maggie Potts and the email address is \JMStaffEmail. This service will be provided for JHU Economic professors only. Recommenders who are not Department of Economics full-time faculty must upload their letters themselves.


\item {\bf \Interfolio ~ (\InterfolioLink):}
Before students start to use the Interfolio System, Maggie must first let students know the type of reference letters that were written by each of their professors for them. For example: Generic, Academic, or Non-Academic. Students need to specify which letter they want uploaded to each application they submit. In other words, students must select a professor and a type of letter the professor has written, each time they submit an application on the Interfolio system.

When entering the recommendation requests in Interfolio please make sure to:
\begin{itemize}
\item Enter the name of your recommender and econ@jhu.edu as their e-mail.
\item For each recommendation request, specify under ``Document Title'' the recommender's name and the type of the recommendation letter (generic, academic, non-academic. etc.) per Maggie's e-mail.
For example, if Maggie emailed you that Professor Ball wrote you both academic and non-academic letters of recommendation, one of your document titles would be: \texttt{Dr. Laurence Ball\_non-academic recommendation} and another would be \texttt{Dr. Laurence Ball\_academic recommendation}.
\item Under ``Recommendation Destination'' there are two choices: General and Specific position. Only choose ``Specific position'' and enter the recommender name along with the type of the reference letter.
\end{itemize}

All Interfolio reference letters will be uploaded to Interfolio directly by Maggie, including the ones from external recommenders.

\item {\bf Pull email}: Some employers have set up systems that allow
  recommenders to upload letters directly themselves.  These employers
  will ask the student to provide the email address of each person who
  is to provide a letter.  The employer then sends an email to each of
  those email addresses, requesting that the letter be uploaded.
  (This is a ``pull'' system because the employer is trying to
  ``pull'' the letter from the recommender.)

  In order for us to keep track of letters in a centralized way, and
  to relieve recommenders of the burden of figuring out how to
  upload their letters, our procedure is as follows.  When the
  employer asks for the email address of a JHU faculty recommender, you should
  always reply with \JMStaffEmail~rather than the faculty member's
  actual email.

  If the recommender is not a JHU faculty member but they would like \JMStaff~to handle uploading their letters, you should just use \JMStaffEmail~as the recommender's email address.  If the recommender {\it wants} to handle their letters themselves, then you can give the employer their real email address.  But this is discouraged, because it means that we do not have any way to track whether the letters have been sent or not.

\item {\bf Push email:} Some employers just provide an email address
  to which letters should be sent.  In order to speed up the process,
  students will provide the staff with as few as possible email groups
  (in general two, for academic and non-academic employers). For push email, we do not 
make any distinction between JHU faculty recommenders and outside recommenders; \JMStaff~will
simultaneously send out letters of both kinds of recommenders using the procedure below.  Students will send to \JMStaffEmail~ one single email
  with subject ``email lists for recommendation letters for [your
  name]'' whose content will look like this: \small
\begin{itemize}
\item {\bf  GROUP 1} (recommenders: Prof.\ Carroll, etc.)
\begin{itemize}
\item {\bf  subject:} Academic Recommendation Letters Needed for [your name]
%\item {\bf  bcc:} CarrollCDJHUEconJobMarket@gmail.com, SecondRecommenderJHUEconJobMarket@gmail.com
\item {\bf  Body:} employer1@aaa.aaa; employer2@aaa.aaa; employer3@aaa.aaa; etc.
\end{itemize}

\item {\bf  GROUP 2} (recommenders: Jesus H. Christ, Carl Christ, Adrian Pagan, etc.)
\begin{itemize}
\item {\bf  subject:} Nonacademic Recommendation Letters Needed for [your name]
%\item {\bf  bcc:} ChristJHJHUEconJobMarket@gmail.com, ChristCJHUEconJobMarket@gmail.com
\item {\bf  Body:} employer1@aaa.aaa; employer2@aaa.aaa; employer3@aaa.aaa; etc.
\end{itemize}

\item {\bf  GROUP 3} (recommenders: James Bond, , etc.)
\begin{itemize}
\item {\bf  subject:} Extra super secret special jobs Recommendation Letters for [your name]
%\item {\bf  bcc:} BondJamesBondJHUEconJobMarket@gmail.com, ...
\item {\bf  Body:} employer1@aaa.aaa; employer2@aaa.aaa; employer3@aaa.aaa; ...
\end{itemize}
\end{itemize}

\textbf{NOTE:} \textit{All email addresses included in the groups must be separated by a semicolon.}

\normalsize

The staff should create one email for each group (copying and pasting
the ``Subject'' fields from the student's email), attach
the appropriate letters (academic or nonacademic) and, in the email
software, make sure to request an ``acknowledged'' receipt.  If error
messages are returned (for example for a mistyped email address, or
over quota of the recipient's account, etc.), the staff will forward
the error message (but not the recommendation letter!) to the student
in question. The student must determine the reason for the error and
provide the staff with an alternative email address.

\begin{comment}
\item {\bf Snail Mail.}
Students will ask the staff for the
  appropriate number of business size envelopes (with the Johns
  Hopkins return address), attach the address labels for their
  employers (the address labels are created by a special macro in the
  spreadsheet -- see
  \url{http://econ.jhu.edu/jobmarket/Information/}), and return
  them divided into 5 groups (academic-USA, non-academic-USA,
  academic-International, non-academic-International) held together by
  rubber bands (one envelope per employer). On the first envelope of
  each group, students will attach a post-it reporting their name, the
  letter type (academic or non-academic), destination (USA or
  international), the names of the recommenders and the number of
  envelopes. Here is an example of the post-it:

\begin{center}
\begin{tabular}{c}
Student's name
\\ Academic - USA
\\ Prof. Carroll
\\ Prof. Faust
\\ Prof. Jeanne
\\ 35 envelopes
\end{tabular}
\end{center}

Staff can thus pick up a group of letters and immediately see how many
copies of the letters to print, the recommenders' names, and the letter
type (academic or non-academic). Staff can ask students to help
sealing the envelopes and applying postage.  The mail is picked up at
1:30 in the afternoon, so the staff will usually need at least a
couple of days to get your letters out.  Further advice from several
former Hopkins students (successful job seekers!) is available in the
documents labelled \texttt{AdviceMoniker.pdf} in the Resources folder
on the \texttt{JHUJobMarket} page.
\end{comment}
\end{enumerate}

\begin{comment}
In sum, both inside and outside recommenders send their letters to \JMStaffEmail.  Details are:

\begin{quote}
\begin{itemize}
\item[EJM] New inside recommenders should set \JMStaffEmail~as their proxy (old inside recommenders should already have done this in prior years, and need to do nothing at all beyond sending letters to \JMStaffEmail). Maggie can only proxy JHU Economics Faculty letters. \textcolor{red}{Recommenders who are not Department of Economics full-time faculty must upload their letters themselves.}
\textcolor{red}{\item[AJO] New inside recommenders should set \JMStaffEmail~as their proxy (old inside recommenders should already have done this in prior years, and need to do nothing at all beyond sending letters to \JMStaffEmail). Maggie can only proxy JHU Economics Faculty letters.} \textcolor{red}{Recommenders who are not Department of Economics full-time faculty must upload their letters themselves.}
\textcolor{red}{\item[AEA/JOE] All inside recommenders should set \JMStaffEmail~as their proxy. The faculty need to set Maggie Potts as their proxy each year, as this system does not carry their proxies over from year to year. Maggie can only proxy JHU Economics Faculty letters.} \textcolor{red}{Recommenders who are not Department of Economics full-time faculty must upload their letters themselves.}
\textcolor{red}{\item[Interfolio]  Inside and outside recommenders follow the same procedures, which require the student to send carefully-constructed emails to \JMStaffEmail~as described above.}
\item[Push Email]  Inside and outside recommenders follow the same procedures, which require the student to send carefully-constructed emails to \JMStaffEmail~as described above.
\item[Pull Email]  For inside recommenders, student requests employer send email to \JMStaffEmail.  For outside recommenders, student requests employer send email to recommender's outside email address.
%\item[Snail Mail]  Inside and outside recommenders follow same procedures, described above.
\end{itemize}

\end{quote}

\end{comment}

\section{Information for Staff}\ifdvi\hypertarget{StaffInfo}{(StaffInfo)}\fi

Most of what the staff will do for recommendation letters is implicit in the previous section.  To
summarize:
\begin{enumerate}
\item You will receive an \EM~spreadsheet from a student.  When you receive it, insert a new Column A and save the spreadsheet to the students electronic folder located on the (X) drive or Stella as it is named. This new column will be used to mark off each employer once their letters are sent to them.
\item {\it After} you receive that spreadsheet, you should start receiving requests for letters through~\JMStaffEmail.  
\item Some naughty students may request letters via~\EJM~or the push or pull email procedures before they send you the spreadsheet; if you receive such requests before having received the student's spreadsheet, please send them a message asking them to send you their spreadsheet, so that you can keep track of which letters have been sent.
\item Whenever you respond to a request for letters through~\JMStaffEmail, put a checkmark on the corresponding line in the student's printed spreadsheet.
\item If a deadline is looming and you do NOT have a checkmark on the student's spreadsheet, there may be a problem of some kind: The employer mistyped the email address for \JMStaffEmail, the student made a mistake somewhere, etc.  As the big deadlines approach, please look at the spreadsheets and see whether there are letters that should have been sent but have not been.
\item If different staff persons are handling different categories of letters (like, one person is handling snail mail letters and another is handling the rest), then each person responsible for a particular category of letters should print their own copy of the spreadsheet and keep track of letters sent.
\end{enumerate}

\section{Information for Faculty}\ifdvi\hypertarget{FacInfo}{(FacInfo)}\fi

\begin{enumerate}

\item {\bf \EJM:}
Each JHU faculty member sending a recommendation letter will need to
get an account at \EJM ~ (\EJMLink). According to Nina, your account is first established
by the STUDENT who first requests a letter from you.  That student must 
provide to \EJM~a valid \texttt{@jhu.edu} email address for you.  Personal email addresses like
those from \texttt{gmail} or \texttt{yahoo} are now prohibited by \EJM~
for security reasons.

If you are not yet registered on~\EJM~and a student provides \EJM~with
a valid JHU email address for you inside \EJM, that will become your
unique~\EJM~email address.  (If you can't remember if you are registered 
but you think you know what email address
you would be registered under if you are,  you can check whether you are
registered by going to \EJM~and trying to log in using that email
address.  If you don't remember the password, you can reset it.)  

So, if you have a preferred JHU email address (JHED actually automatically
sets up a number of aliases for each faculty member), you need to tell
your students what that address is.  (They should ask; but it can't
hurt to proactively tell them).

Once you have been registered for \EJM~and received a confirmation
email to your JHU account, you can designate a ``proxy'' to handle the
actual work of uploading your letters.  We (the Job Market Collective
Front) insist that you do things this way, because that provides us (the Front)
with a centralized way of keeping track of where the process is.  To
repeat, you must NOT upload your letters yourself; you MUST designate
a proxy.

The proxy is Maggie Potts; the email address to use for the proxy is
\JMStaffEmail.  The good thing about using a proxy is that you (as a faculty member)
need only to send your letters out once, to \JMStaffEmail, and everything 
else is handled for you after that.  

\begin{comment}
\textcolor{red}{
\item {\bf \AJO:}
Each JHU faculty member must register at \AJOLink~ and enter Maggie Potts as their proxy, using the
\JMStaffEmail~ email address. To do this, log in to your account and click on the `proxy' link near the bottom. This only needs to be done once and the proxy works for all applicants the faculty may have on the job market that year.}
\end{comment}

\item {\bf \AEA:}
Each faculty member must register at \AEARecLink. After you create an account, you will have to designate Maggie as a surrogate to manage your reference letters. The surrogate name is Maggie Potts and the email address is \JMStaffEmail. The faculty need to set Maggie Potts as their proxy each year, as this system does not carry their proxies over from year to year.

\end{enumerate}

N.B.:  When you send your letters to \JMStaffEmail, please use self-explanatory names, like
\texttt{CarrollCD-For-WhiteMN-Academic.pdf}, or
\texttt{CarrollCD-For-WhiteMN-NonAcademic.pdf}, or, if the same letter
is to be sent both to academic and to nonacademic employers,
\texttt{CarrollCD-For-WhiteMN-Generic.pdf}.

\end{document}
